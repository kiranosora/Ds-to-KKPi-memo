\section{Event Selection}
\label{ST-selection}
At $E_{cm} = 4.178$ GeV, pairs of $D_{s}D_{s}^{*}$ are produced,  and the $D_{s}^{*}$ decays to either $D_{s}\gamma$ or $D_{s}\pi^{0}$.
So the $D_{s}$ mesons are produced in pairs without additional charged hadrons ($K^{\pm}(\pi^{\pm})$).
We reconstruct the gamma from the $D_{s}^{*}$ decay in this analysis except for the branching fraction measurement section (Sec.~\ref{BFM}). 
These unique $D_{s}^{+}D_{s}^{-}$ final states provide us an opportunity to employ the double tag method to measure the absolute branching fractions of $D_{s}$ meson decays.
The double tag method also provides clean samples to perform an amplitude analysis.


\subsection{Tracking, PID, $\pi^{0}$, $\eta$, $\eta^{'}$ and $K_{S}^{0}$ Reconstruction }
$D_{s}$ candidates are built from $K^{\pm}$, $\pi^{\pm}$, $\pi^{0}$, $\eta$, $\eta^{'}$ and $K_{S}^{0}$. The selections of the particles to build $D_{s}$ candidates are performed with DTagAlg-00-01-05 package with the default setting, which are summarized below.

\begin{itemize}
	\item Tracking:
		\begin{itemize}
			\item[-] The properties of charged tracks are determined based on the MDC information. Charged track candidates must satisfy:
				\begin{itemize}
					\item[-] $|cos\theta| < 0.93$, 
                    \item[-] $|dr| < \boldmath 1 $ cm and $|dz| < \boldmath 10$ cm,

						where $|dr|$ and $|dz|$ are distances of the track from the beam position in x-y plane and z plane, respectively.
				\end{itemize}
		\end{itemize}
	\item Particle ID:
		\begin{itemize}
			\item[-] Charged tracks are identified as pions or kaons with Particle Identification (PID), which is implemented by combing the information of the energy loss (dE/dx) in MDC and the time-of-flight measured from
				the TOF system. Kaon and Pion are identified with the requirements that 
				\begin{itemize}
					\item[-] $\emph{Prob(K)} > 0$ and $\emph{Prob(K)} > \emph{Prob($\pi$)}$ for \emph{K},
					\item[-] $\emph{Prob($\pi$)} > 0$ and $\emph{Prob($\pi$)} > \emph{Prob(K)}$ for \emph{$\pi$},
						
                        where \emph{Prob(X)} is the probability of hypothesis X, X can be \emph{$\pi$} or \emph{K}.
				\end{itemize}
		\end{itemize}
    \item $\pi^{0}/\eta$ selection: $\pi^{0}/\eta$ candidates are reconstructed through $\pi^{0} \rightarrow \gamma\gamma\ (\eta \rightarrow \gamma\gamma)$ with package of PioEtaToGGRecAlg.
		

		The photons are reconstructed as energy showers on the EMC. We require:
		\begin{itemize}
			\item[-] Minimum energy for barrel showers ( $|cos\theta| < 0.8$): $E_{min} >25$ MeV/$c^{2}$,
			\item[-] Minimum energy for endcap showers ( $0.86 < |cos\theta| < 0.92$): $E_{min} >50$ MeV$/c^{2}$,
			\item[-] Shower within other $|cos\theta|$ regions are rejected,
			\item[-] The shower time is required to be within 700 ns of the event start time to suppress the electronics noise. 
		\end{itemize}

		Then we perform a constrained fit on the photon pairs to the nominal $\pi^{0}/\eta$ mass and require:
		\begin{itemize}
            \item[-] The unconstrained invariant mass for $\pi^{0}$: $0.115 < M(\gamma\gamma) < 0.150\ {\rm GeV}/c^{2}$,
            \item[-] The unconstrained invariant mass for $\eta: 0.490 < M(\eta) < 0.580\ {\rm GeV}/c^{2}$,
			\item[-] Mass fit: $\chi_{1c}^{2} < 30$.
		\end{itemize}
	\item $\eta^{'}$ selection: The $\eta^{'}$ candidates are reconstructed with $\pi^{+}\pi^{-}\eta$, the invariant mass for $\pi^{+}\pi^{-}\eta$ is required to fall into the range of $[0.938, 0.978]$ GeV$/c^{2}$.
	\item $K_{S}^{0}$ selection: $K_{S}^{0}$ candidates are reconstructed using VeeVertexAlg package with two opposite charged tracks with requiring:
		\begin{itemize}
			\item[-] $|cos\theta| < 0.93$,
			\item[-] $|dz| < 20$ cm.
		\end{itemize}

		For each pair of tracks, a constrained vertex fit is performed and the track parameters' results are used to get the invariant mass $M(K_{S}^{0})$. Then the decay length of $K_{S}^{0}$ is obtained with second vertex 
		fit by the SecondVertexFit package. For $K_{S}^{0}$ selection, we require:
		\begin{itemize}
            \item[-] ${\rm 0.487\ GeV}/c^{2} < M(K_S^{0}) < 0.511\ {\rm GeV}/c^{2}$,
			\item[-] Significance of the $K_{S}^{0}$ decay length is at least two standard deviations.
		\end{itemize}

\end{itemize}
\subsection{$D_{s}$ Selection}
The $D_{s}$ candidates are constructed from individual $\pi$, K, $\eta$, $\eta^{'}$, $K_{S}^{0}$ and $\pi^{0}$ in an event.
The $D_{s}$ candidates fall into the mass window of $1.87 < M(D_{s}) < 2.06\ {\rm GeV}/c^{2}$ and 
the corresponding $M_{rec}$ satisfied $2.051 < M_{rec} < 2.180\ {\rm GeV}/c^{2}$ are retained for further study. The definition of $M_{rec}$ is
\begin{equation}
    M_{rec} = \sqrt{(E_{cm} - \sqrt{p_{D_{s}}^{2} + m_{D_{s}}^{2}})^{2} - |\vec p_{cm} - \vec p_{D_{s}} | ^{2}} \; , \label{con:inventoryflow}
\end{equation}
where $E_{cm}$ is the energy of initial state calculated from the beam energy~\cite{DocDB 580-v1}, $p_{D_{s}}$ is the momentum of $D_{s}$ candidate, $m_{D_{s}}$ is $D_{s}$ mass quoted from PDG~\cite{PDG2018}, and $\vec p_{cm}$ and $\vec p_{D_{s}}$ are four-momentum of the initial state and the decay products of the $D_{s}$ candidate, respectively.

\subsection{Signal Selection}
\par{


    We use different methods to select events in the model independent analysis (Sec.~\ref{MIPWASelection}), the amplitude analysis (Sec.~\ref{AASelection}) and the branching fraction measurement (Sec.~\ref{BFSelection}).
%And the details are in the corresponding sections.
}


%%%%%%%%%%%%%%%%%%%%%%%%%%%%


%%%%%%%%%%%%%%%%%%%%%%%%%%%%
