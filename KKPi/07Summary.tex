\section{Summary}
\par{
    This analysis presents the amplitude analysis for the decay $D_{s}^{+} \rightarrow K^{+}K^{-}\pi^{+}$.
    Below is a comparison of amplitude analysis between BABAR, CLEO-c and this analysis. Our results are roughly consistent with those of BABAR and CLEO-c.
    And for the fit fraction of $D_{s}^{+} \rightarrow f_{0}(980)\pi^{+}/a_{0}(980)\pi^{+}$, we tend to agree the result of BABAR.
    \begin{table}
        \caption{Comparision between BABAR, CLEO-c and this amplitude analysis.}
        \label{final-result}
        \begin{center}
            \begin{tabular}{cccc}
                \toprule\toprule
                Amplitude & BABAR  & CLEO-c  & This Analysis\\
                \hline
                $D_{s}^{+} \rightarrow \bar{K}^{*}(892)^{0}K^{+}$              & 47.9$\pm$0.5$\pm$0.5  & 47.4$\pm$1.5$\pm$0.4& 48.3$\pm$0.9$\pm$0.4 \\
                $D_{s}^{+} \rightarrow \phi(1020)\pi^{+}$                      & 41.4$\pm$0.8$\pm$0.5  & 42.2$\pm$1.6$\pm$0.3& 40.5$\pm$0.7$\pm$0.9 \\
                $D_{s}^{+} \rightarrow f_{0}(980)\pi^{+}/a_{0}(980)\pi^{+}$    & 16.4$\pm$0.7$\pm$2.0  & 28.2$\pm$1.9$\pm$1.8& 19.3$\pm$1.7$\pm$2.0 \\
                $D_{s}^{+} \rightarrow \bar{K}^{*}_{0}(1430)^{0}K^{+}$         & 2.4$\pm$0.3$\pm$1.0   & 3.9$\pm$0.5$\pm$0.5 & 3.0$\pm$0.6$\pm$0.5  \\
                $D_{s}^{+} \rightarrow f_{0}(1710)\pi^{+}$                     & 1.1$\pm$0.1$\pm$0.1   & 3.4$\pm$0.5$\pm$0.3 & 1.9$\pm$0.4$\pm$0.6  \\
                $D_{s}^{+} \rightarrow f_{0}(1370)\pi^{+}$                     & 1.1$\pm$0.1$\pm$0.2   & 4.3$\pm$0.6$\pm$0.5 & 1.2$\pm$0.4$\pm$0.2  \\
                $\begin{matrix}\sum FF(\%)\end{matrix}$                          & 110.2$\pm$0.6$\pm$2.0 & 129.5$\pm$4.4$\pm$2.0 & 114.2$\pm$1.7$\pm$2.3\\
                $\chi^{2}/NDF$                                                  & $\frac{2843}{2305-14}=1.2$ & $\frac{178}{117}=1.5$ & $\frac{290}{291-10-1}=1.04$\\
                Events                                                         &$96307\pm369$(purity$\ 95\%$)          &$14400$(purity$\ 85\%$)  &$4381$(purity$\ 99.7\%$)\\
                \bottomrule\bottomrule
            \end{tabular}
        \end{center}
    \end{table}
    %And for the fit fraction of $D_{s}^{+} \rightarrow f_{0}(980)\pi^{+}/a_{0}(980)\pi^{+}$, the result is roughly consistent with the previous analyses and we tend to agree the result of BABAR.

    %\begin{table}
    %    \caption{The result of this analysis.}
    %    \label{final-conclusion}
    %    \begin{center}
    %        \begin{tabular}{cccc}
    %            \toprule
    %            Amplitude & Magnitude  & Phase  & Fit fractions(\%)\\
    %            \hline
    %            $D_{s}^{+} \rightarrow \bar{K}^{*}(892)^{0}K^{+}$              & 1.0(fixed)             & 0.0(fixed)                & 48.3$\pm$0.9$\pm$0.4\\
    %            $D_{s}^{+} \rightarrow \phi(1020)\pi^{+}$                      & 1.09$\pm$0.02$\pm$0.01 & 6.22$\pm$0.07$\pm$0.04    & 40.5$\pm$0.7$\pm$0.9\\
    %            $D_{s}^{+} \rightarrow f_{0}(980)\pi^{+}/a_{0}(980)\pi^{+}$    & 2.88$\pm$0.14$\pm$0.16 & 4.77$\pm$0.07$\pm$0.07    & 19.3$\pm$1.7$\pm$2.0\\
    %            $D_{s}^{+} \rightarrow \bar{K}^{*}_{0}(1430)^{0}K^{+}$         & 1.26$\pm$0.14$\pm$0.15 & 2.91$\pm$0.20$\pm$0.23    & 3.0$\pm$0.6$\pm$0.5\\
    %            $D_{s}^{+} \rightarrow f_{0}(1710)\pi^{+}$                     & 0.79$\pm$0.08$\pm$0.14 & 1.02$\pm$0.12$\pm$0.05    & 1.9$\pm$0.4$\pm$0.6\\
    %            $D_{s}^{+} \rightarrow f_{0}(1370)\pi^{+}$                     & 0.58$\pm$0.08$\pm$0.08 & 0.59$\pm$0.17$\pm$0.45    & 1.2$\pm$0.4$\pm$0.2\\
    %            \bottomrule
    %        \end{tabular}
    %    \end{center}
    %\end{table}
    In this analysis, as $a_{0}(980)$ and $f_{0}(980)$ are close to each other and parameters of $f_{0}(980)$ is not well measured, we have to take $f_{0}(980)$ and $a_{0}(980)$ as a whole, that is the $\mathcal{S}$ wave at the low end of $K^{+}K^{-}$ mass spectrum. 
    The $\mathcal{S}$ wave is extracted with the model independent method.
    
   
    
    We also measured the branching farction and the value is $\mathcal{B}(D_{s}^{+} \rightarrow K^{+}K^{-}\pi^{+})=(5.47\pm0.07_{stat.}\pm0.11_{sys.})\%$.
    As is shown in Table \ref{BF-Compare}, the branching fraction of this analysis has the best precision.
    \begin{table}
        \caption{Comparisions of branching fractiong between BABAR, CLEO-c and this analysis.}
        \label{BF-Compare}
        \begin{center}
            \begin{tabular}{cc}
                \toprule\toprule
                $\mathcal{B}$ $(D_{s}^{+} \rightarrow K^{+}K^{-}\pi^{+})(\%)$ & TECN  \\
                \hline
                $5.55\pm0.14_{stat.}\pm0.13_{sys.}$    &  CLEO-c  ~\cite{CLEO-BF}                     \\
                $5.06\pm0.15_{stat.}\pm0.21_{sys.}$    &  BELLE ~\cite{BELL-BF}                     \\
                $5.78\pm0.20_{stat.}\pm0.30_{sys.}$    &  BABAR ~\cite{BABAR-BF}                    \\
                $5.47\pm0.07_{stat.}\pm0.11_{sys.}$                     &  This Analysis(BESIII)    \\
                %$\mathcal{B}(D_{s}^{+} \rightarrow K^{+}K^{-}\pi^{+})(\%)$ &   $5.47\pm0.07_{stat.}\pm0.11_{sys.}$ &   $5.55\pm0.14_{stat.}\pm0.13_{sys.}$ &   $5.47\pm0.07_{stat.}\pm0.11_{sys.}$ 
                \bottomrule\bottomrule
            \end{tabular}
        \end{center}
    \end{table}

    %This analysis will be used to do measuarement of absolute hadronic branching fractions of $D_{s}$ meson.
}
%%%%%%%%%%%%%%%%%%%%%%%%%%%%


%%%%%%%%%%%%%%%%%%%%%%%%%%%%
