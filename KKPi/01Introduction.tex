\section{Introduction}

\par{The decay $D_{s}^{+} \rightarrow K^{+}K^{-}\pi^{+}$ is a Cabibbo-favored (CF) channel and has a large branching fraction for the $D_{s}$ meson. Thus, this decay channel is usually used to normalize measurements of decay chains involving charm quarks.

Knowledge of the decay amplitude allows us to properly account for interference effects when measuring absolute hadronic branching fractions of $D_{s}$ mesons.

Below is the table of previous analyses of this decay channel.}

\begin{figure*}[htbp]
 \centering
 \mbox{
  %\vskip -1.5cm
  \begin{overpic}[width=0.96\textwidth, height=0.4\textwidth]{plot/PreviousAnalyses.png}
  \end{overpic}
 }
\caption{previous analyses }
\label{PreviousAnalyses}
\end{figure*}


From Fig.\ref{PreviousAnalyses}~\cite{2011BARBAR}, we can see an obvious difference of decay fraction of $f_{0}(980)\pi^{+}$ between BARBAR and CLEO-c. E687 used about 700 events and the E687 model did not take $f_{0}(1370)\pi^{+}$ into account. For CLEO-c, about 14400 events with purity about 84.9\% were selected with single tag method. The analysis of BARBAR used about 100000 events with purity about 95\%. In this analysis with double tag method, we can get a nearly background free data sample, that will be good to perform the amplitude analysis.

\iffalse
As shown in Figure~\ref{fig:lambc_cs} and Figure~\ref{fig:lambc_cs_bes3}, at the energy of 4.6\,GeV, cross section of producing $\lambdacp\lambdacm$ pair in $\ee$ collisions is $\sigma(\ee\to\lambdacp\lambdacm)=0.38\pm0.13\,\rm{nb}$ measured by BELLE~\cite{Pakhlova:2008vn} and $\sigma(\ee\to\lambdacp\lambdacm)=0.253\pm0.023\,\rm{nb}$ measured by BESIII~\cite{Weiping:lineshape}.\\

%%%%%%%%%%%%%%%%%%%%%%%%%%%%


%%%%%%%%%%%%%%%%%%%%%%%%%%%%
\begin{figure*}[h]
\centering
\includegraphics[width=0.45\textwidth]{bes3_lineshape.eps}
\caption{Cross sections of $\ee\to\lambdacp\lambdacm$ measured by BESIII.}
\label{fig:lambc_cs_bes3}
\end{figure*}
%%%%%%%%%%%%%%%%%%%%%%%%%%%%%
\fi
