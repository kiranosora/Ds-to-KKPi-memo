\section{Branching Fraction Measurements}

\subsection{Event Selection}
\label{BFSelection}
After the selection described in Set.\ref{ST-selection}, we continue to select signals used in this part. 
We use the same 8 tag modes as in Sec.\ref{AASelection}.
In the selection of tag $D_{s}$, for multiple candidates, the best candidate is chosen with $M_{rec}$ closest to mass of $D_{s}^{*}$ in ~\cite{PDG2018}.
To further remove the background associated with the larger number of soft $\pi^{\pm}(\pi^{0})$ from $D_{s}^{*}$ decays, candidates are voted if the momentum of $\pi^{\pm}(\pi^{0})$ is less than $0.1GeV$.

The single tag(ST) yields are extracted from the fits to the $D_{s}$ invariant mass distributions, as shown in Fig.\ref{SingleTagFit}. In the fit, the mass windows of the tag modes are set to be the same as the Ref. ~\cite{Doc-DB-630-v35}.
The signal shape is modeled as MC shape convoluted with a Gaussian function, whie background is parameterized as the 2nd-order Chebychev polynomial.
The corresponding ST effeciencies are estimated from generic MC. The ST yields($Y_{ST}$) and ST efficiencies($\epsilon_{ST}$) are listed in Table\ref{ST-eff}

\begin{table}
    \caption{ The ST yields($Y_{ST}$) and ST efficiencies($\epsilon_{ST}$). The mass windows use the results in Ref. ~\cite{Doc-DB-630-v35} }
    \label{ST-eff}
    \begin{center}
        \begin{tabular}{cccc}
            \toprule\toprule
            Tag mode & Mass window(GeV)  & $Y_{ST}$  & $\epsilon_{ST}(\%)$\\
            \hline
            $D_{s}^{-} \rightarrow K_{S}^{0}K^{-}$                          & [1.948, 1.991]    & $965265\pm1286$               & 49.48$\pm$0.07\\
            $D_{s}^{-} \rightarrow K^{+}K^{-}\pi^{-}$                       & [1.900, 2.030]    & $4254481\pm2947$              & 42.17$\pm$0.03\\
            $D_{s}^{-} \rightarrow K^{+}K^{-}\pi^{-}\pi^{0}$                & [1.947, 1.982]    & $1161036\pm3400$              & 10.61$\pm$0.03\\
            $D_{s}^{-} \rightarrow K_{S}^{0}K^{-}\pi^{+}\pi^{-}$            & [1.958, 1.980]    & $233225\pm1467$               & 19.30$\pm$0.12\\
            $D_{s}^{-} \rightarrow K_{S}^{0}K^{+}\pi^{-}\pi^{-}$            & [1.953, 1.983]    & $484801\pm1312$               & 22.72$\pm$0.06\\
            $D_{s}^{-} \rightarrow \pi^{-}\pi^{-}\pi^{+}$                   & [1.952, 1.984]    & $1152623\pm3370$              & 56.94$\pm$0.17\\
            $D_{s}^{-} \rightarrow \pi^{-}\eta_{\pi^{+}\pi^{-}\eta_{\gamma\gamma}}^{'}$          & [1.940, 1.996]    & $251617\pm737$               & 20.43$\pm$0.06\\
            $D_{s}^{-} \rightarrow K^{-}\pi^{+}\pi^{-}$                     & [1.953, 1.983]    & $610925\pm2902$               & 47.18$\pm$0.22\\
            \bottomrule\bottomrule
        \end{tabular}
    \end{center}
\end{table}

\begin{figure*}[!htbp]
 \centering
 \includegraphics[width=0.4\textwidth]{plot/DsTag400_Mass_data.eps}
 \includegraphics[width=0.4\textwidth]{plot/DsTag401_Mass_data.eps}
 \includegraphics[width=0.4\textwidth]{plot/DsTag404_Mass_data.eps}
 \includegraphics[width=0.4\textwidth]{plot/DsTag405_Mass_data.eps}
 \includegraphics[width=0.4\textwidth]{plot/DsTag406_Mass_data.eps}
 \includegraphics[width=0.4\textwidth]{plot/DsTag421_Mass_data.eps}
 \includegraphics[width=0.4\textwidth]{plot/DsTag460_Mass_data.eps}
 \includegraphics[width=0.4\textwidth]{plot/DsTag502_Mass_data.eps}
 \caption{Ds Mass fits from data. The points with error bars are data, and the blue line is the fit. Red short-dashed lines are signal, violet long-dashed lines are background. The red arrows denote the signal region.  }
\label{SingleTagFit}
\end{figure*}


After selected the candidates of tag $D_{s}$, for each tag mode, we can get double tag candidates. %We retain the candidates with minimum average mass ($aM$) of tag $D_{s}$ and signal $D_{s}$.


With the updated MC sample(DIY MC) based on the amplitude analysis results obtained from the fit to data, the double efficiencies are determined and listed in Table\ref{DT-eff}.

    \begin{table}
        \caption{ The DT efficiencies($\epsilon_{DT}$).}
        \label{DT-eff}
        \begin{center}
            \begin{tabular}{cccc}
                \toprule\toprule
                Tag mode   & $\epsilon_{DT}(\%)$\\
                \hline
                $D_{s}^{-} \rightarrow K_{S}^{0}K^{-}$                                                   & 19.77$\pm$0.14\\
                $D_{s}^{-} \rightarrow K^{+}K^{-}\pi^{-}$                                                & 18.21$\pm$0.06\\
                $D_{s}^{-} \rightarrow K^{+}K^{-}\pi^{-}\pi^{0}$                                         &  4.69$\pm$0.03\\
                $D_{s}^{-} \rightarrow K_{S}^{0}K^{-}\pi^{+}\pi^{-}$                                     &  8.34$\pm$0.11\\
                $D_{s}^{-} \rightarrow K_{S}^{0}K^{+}\pi^{-}\pi^{-}$                                     &  9.55$\pm$0.09\\
                $D_{s}^{-} \rightarrow \pi^{-}\pi^{-}\pi^{+}$                                            & 23.72$\pm$0.15\\
                $D_{s}^{-} \rightarrow \pi^{-}\eta_{\pi^{+}pi^{-}\eta_{\gamma\gamma}}^{'}$               &  8.70$\pm$0.11\\
                $D_{s}^{-} \rightarrow K^{-}\pi^{+}\pi^{-}$                                              & 19.68$\pm$0.17\\
                \bottomrule\bottomrule
            \end{tabular}
        \end{center}
    \end{table}

\par{
    \subsection{Analytic Strategy}
    The data sample for this analysis is collected at $E_{cm}=4.178GeV$, about $100MeV$ higher than $D_{s}^{*}D_{s}$ threshold. Around this energy region, based on the cross section measurement by CLEO ~\cite{PRD80-072001}, we know that most $D_{s}$ productionin $e^{+}e^{-}$ collision comes from $D_{s}^{*\pm}D_{s}^{\mp}$ events with its cross section aprroximatedly $0.9nb$, while the cross section for $D_{s}^{+}D_{s}^{-}$ is about a factor of 20 smaller.
    The $D_{s}^{*}$ decays to either $\gamma D_{s}$ or $\pi^{0}D_{s}$ with branching fractions of $(93.5\pm0.7)\%$ and $(5.8\pm0.7)\%$ ~\cite{PDG2018}, respectively. 
    The other charm productions total $~7pb$ mainly including $D^{*}\bar{D}^{*}$ with a cross section of $~5nb$, $D^{*}\bar{D} + D\bar{D}^{*}$ with a cross section of $~2nb$, and $D\bar{D}$ with a cross sectionof relatively small $~0.2 nb$.
    There also appears to be $D\bar{D}^{*}\pi$ production.
    The underlying light quark "continuum" background is about $12 nb$. 
    The relatively large cross sections, relatively large branching fractions, and sufficient luminosities allow us to employ double tag (DT) technique to study this study.

    As $D_{s}^{-} \rightarrow K^{+}K^{-}\pi^{-}$ is not only our signal mode but also one of our tag modes, we divide the events into two categories:

    \begin{itemize}
        \item[-] Cat. A: Tag $D_{s}$ decays to tag modes except $D_{s}^{-} \rightarrow K^{+}K^{-}\pi^{-}$. The generic MC sample with the signal removed shows no peaking background around the fit range of $1.90 < M_{sig} < 2.03 GeV$.
            Thus, the double tag yield is determined by the fit to $M_{sig}$, which has shown in Fig.\ref{DT-fit}(a).And the background is described with $2^{nd}$-order Chebychev polynomial. The double tag yield is $3484\pm64$. 
        \item[-] Cat. B: Tag $D_{s}$ decays $K^{+}K^{-}\pi^{+}$. As both of the two $D_{s}$ mesons decay to our signal modes, we fit $aM$(the average mass of $D_{s}$ at signal side and tag side), which is shown in Fig.\ref{DT-fit}(b). 
            Here, the background is described with $2^{nd}$-order Chebychev polynomial. The double tag yield is $1651\pm42$. 
    \end{itemize}

    \begin{figure*}[!htbp]
        \centering
        \includegraphics[width=0.4\textwidth]{plot/DT-A.eps}
        \includegraphics[width=0.4\textwidth]{plot/DT-B.eps}
        \caption{Fit of (a)Cat. A and (b)Cat. B.
            We fit $M_{sig}$ and $aM$ for Cat. A and Cat. B, respectively. The signal shapes are the corresponding simulated shapes convoluted with a Gaussian function and 
        the background shapes are described with $2^{nd}$-order Chebychev polynomial.}
        \label{DT-fit}
    \end{figure*}

    To measure the branching fraction of this decay, we start from the following equations with one tag mode:
    \begin{equation}
        N_{tag}^{obs} = 2N_{D_{s}^{+}D_{s}^{-}}\mathcal{B}_{tag}\epsilon_{tag}, \label{eq-ST}
    \end{equation}

    \begin{equation}
        \begin{array}{lr}
            N_{sig}^{obsA}=2N_{D_{s}^{+}D_{s}^{-}}\mathcal{B}_{tag}\mathcal{B}_{sig}\epsilon_{tag,sig}  , &\text{for Cat. A} \\
            N_{sig}^{obsB}=N_{D_{s}^{+}D_{s}^{-}}\mathcal{B}_{tag}\mathcal{B}_{sig}\epsilon_{tag,sig}  ,  &\text{  for Cat. B}  
        \end{array}
        \label{eq-DT}
    \end{equation}
    where $N_{D_{s}^{+}D_{s}^{-}}$ is the total number of $D_{s}^{*\pm}D_{s}^{\mp}$ produced from $e^{+}e^{-}$ collision; $N_{tag}^{obs}$ is the number of observed tag modes; $N_{sig}^{obsA}$ and $N_{sig}^{obsB}$ are the number of observed signals for Cat. A and Cat. B, respectively; $\mathcal{B}_{tag}$ and $\mathcal{B}_{sig}$ are the branching fractions of specific tag mode and signal mode, respectively; $\epsilon_{tag}$ is the reconstruction efficiency of the tag mode; $\epsilon_{tag,sig}$ is the reconstruction efficiency of both the tag and signal decay modes.

    Using the above equations, it's easy to obtain:
    \begin{equation}
    \mathcal{B}_{sig} = \frac{N_{sig}^{obsA}+2N_{sig}^{obsB}}{\begin{matrix}\sum_{\alpha} N_{tag}^{\alpha}\epsilon_{tag,sig}^{\alpha}/\epsilon_{tag}^{\alpha}\end{matrix}}, \label{BR-formula}
    \end{equation}
    where the yields $N_{tag}^{obsA}$, $N_{tag}^{obsB}$ and $N_{tag}^{\alpha}$ can be obtained from data, while $\epsilon_{tag}$ and $\epsilon_{tag,sig}$ can be obtained from the appropriate MC samples, $\alpha=1, 2, 3...$ represents the tag mode.
    

    \subsection{Results of Branching Fraction}



    Then we can get the branching fraction $\mathcal{B}(D_{s}^{+} \rightarrow K^{+}K^{-}\pi^{+})=(5.47\pm0.07)\%$(only statistical uncertainty) according to Eq.\ref{BR-formula}.

    \subsection{Systematic Uncertainties}
    The following sources are taken in account to calculate systematic uncertainties.

    \begin{itemize}
        \item Signal shape. The systematic uncertainty due to the signal shape is studied with the fit without the Gaussian function convoluted, the double tag yield shift is taken as the related effect. 

        \item Background shape and fit range. For background shape and the fit range, the MC shape is used to instead the $1^{st}$-order Chebychev polynomial and the fit range of $[1.90, 2.03]GeV$ is changed to $[1.89, 2.04]GeV$. 
            The largest branching fraction shift is taken as the related effect.
        
        \item Fit bias. The possible bias is estimated by the IO check using the round 30-40 of generic MC, which is shown in Table \ref{BR-IO}. 
            The estimated mean ($\mu_{\mathcal{B}}$) and its uncertainty ($\sigma_{\mu}$) is calculated with the following formulas:
            \begin{equation}
            \mu_{\mathcal{B}} = \frac{\begin{matrix}\sum_{i}\frac{\mu_{i}}{\sigma_{i}^{2}}\end{matrix}}{\begin{matrix}\sum_{i}\frac{1}{\sigma_{i}^{2}}\end{matrix}}, \ \ \ \ \sigma_{\mu}^{2}=\begin{matrix}\sum_{i}\frac{1}{\sigma_{i}^{2}}\end{matrix},
            \label{BR-Combined}
            \end{equation}
            where $\mu_{i}$ and $\sigma_{i}$ are the measured branching fraction value and its statistial uncertainty for the sample i. The combined result of the round 30-40 is $\mu_{\mathcal{B}} = (5.462 \pm 0.021)\%$. 
            The relative change compared to the input value is $0.1\%$.

            \begin{table}
                \caption{IO check using the round 30-40 of generic MC.}
                \label{BR-IO}
                \begin{center}
                    \begin{tabular}{cccc}
                        \toprule\toprule
                        Round   &$\mathcal{B}(D_{s}^{+} \rightarrow K^{+}K^{-}\pi^{+})$(\%) \\
                        \hline
                        31                                  & $5.562 \pm 0.066$\\ 
                        32                                  & $5.497 \pm 0.066$\\
                        33                                  & $5.407 \pm 0.066$\\
                        34                                  & $5.636 \pm 0.068$\\
                        35                                  & $5.490 \pm 0.066$\\
                        36                                  & $5.397 \pm 0.066$\\
                        37                                  & $5.369 \pm 0.066$\\
                        38                                  & $5.490 \pm 0.067$\\
                        39                                  & $5.353 \pm 0.065$\\
                        40                                  & $5.435 \pm 0.066$\\
                        \hline
                        Combined result                               & $5.462 \pm 0.021$\\
                        \bottomrule\bottomrule
                    \end{tabular}
                \end{center}
            \end{table}

        \item $K^{\pm}$ and $\pi^{\pm}$ Tracking/PID efficiency. Based on the works ~\cite{PID} and ~\cite{Tracking} by Xingyu Shan and Sanqiang Qu, etc, 
            we find that it's enough to assign $1.1\%$ , $0.4\%$, $1.1\%$ and $0.2\%$ as the systematic uncertainty for $K^{\pm}$ PID, $\pi^{\pm}$ PID,  $K^{\pm}$ tracking, $\pi^{\pm}$ tracking efficiencies, respectively.

    \item MC statistics. The uncertainty of MC statistics is obtained by $\sqrt{ \begin{matrix} \sum_{i} f_{i}\frac{\delta_{\epsilon_{i}}}{\epsilon_{i}}i\end{matrix}}$, where $f_{i}$ is the tag yield fraction and $\epsilon_{i}$ is the signal efficiency of tag mode $i$.
    \end{itemize}

    All of the systematic uncertainties mentioned above are summarized in Table \ref{BF-Sys}.
    \begin{table}
        \caption{Systematic uncertainties of branching fraction.}
        \label{BF-Sys}
        \begin{center}
            \begin{tabular}{cccc}
                \toprule\toprule
                Source   & Sys. Uncertainty\\
                \hline
                Signal shale                        & 0.3 \\
                Background shape and fit range      & 1.1 \\
                Fit bias                            & 0.1 \\
                $K^{\pm}$ PID efficiency            & 1.1 \\
                $\pi^{\pm}$ PID efficiency          & 0.4 \\
                $K^{\pm}$ Tracking efficiency       & 1.1 \\
                $\pi^{\pm}$ Tracking efficiency     & 0.2 \\
                MC statistics                       & 0.2 \\
                \hline
                total                               & 2.0 \\
                \bottomrule\bottomrule
            \end{tabular}
        \end{center}
    \end{table}

    The branching farction with systematic uncertainties is $\mathcal{B}(D_{s}^{+} \rightarrow K^{+}K^{-}\pi^{+})=(5.47\pm0.07_{stat.}\pm0.11_{sys.})\%$.




}
