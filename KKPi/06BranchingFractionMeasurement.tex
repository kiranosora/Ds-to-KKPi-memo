\section{Branching Fraction Measurements}

\par{
    \subsection{Results of Branching Fraction}
    With the amplitude analysis results obtained from the fit to data, the double efficiencies without any kinematic fit requirement applied are determined and listed in Tab.\ref{DT-eff}.

    \begin{table}
        \caption{ The DT efficiencies($\epsilon_{DT}$).}
        \label{DT-eff}
        \begin{center}
            \begin{tabular}{cccc}
                \toprule
                Tag mode   & $\epsilon_{DT}(\%)$\\
                \hline
                $D_{s}^{-} \rightarrow K_{S}^{0}K^{-}$                                                   & 19.61$\pm$0.14\\
                $D_{s}^{-} \rightarrow K^{+}K^{-}\pi^{-}$                                                & 18.26$\pm$0.06\\
                $D_{s}^{-} \rightarrow K^{+}K^{-}\pi^{-}\pi^{0}$                                         &  4.68$\pm$0.03\\
                $D_{s}^{-} \rightarrow K_{S}^{0}K^{-}\pi^{+}\pi^{-}$                                     &  8.28$\pm$0.11\\
                $D_{s}^{-} \rightarrow K_{S}^{0}K^{+}\pi^{-}\pi^{-}$                                     &  9.52$\pm$0.09\\
                $D_{s}^{-} \rightarrow \pi^{-}\pi^{-}\pi^{+}$                                            & 23.55$\pm$0.15\\
                $D_{s}^{-} \rightarrow \pi^{-}\eta_{\pi^{+}pi^{-}\eta_{\gamma\gamma}}^{'}$               &  8.28$\pm$0.11\\
                $D_{s}^{-} \rightarrow K^{-}\pi^{+}\pi^{-}$                                              & 20.00$\pm$0.19\\
                \bottomrule
            \end{tabular}
        \end{center}
    \end{table}

    We divide the events into two categories:

    \begin{itemize}
        \item[-] Cat. A: Tag $D_{s}$ decays to tag modes except $D_{s}^{-} \rightarrow K_{S}^{0}K^{-}$. The generic MC sample with the signal removed shows no peaking background around the fit range of $1.90 < M_{sig} < 2.03 GeV$.
            Thus, the double tag yield is determined by the fit to $M_{sig}$, which has shown in Fig.\ref{DT-fit}(a).And the background is described with $2^{nd}$-order Chebychev polynomial. The double tag yield is $3503\pm64$. 
        \item[-] Cat. B: Tag $D_{s}$ decays $K^{+}K^{-}\pi^{+}$. As both of the two $D_{s}$ mesons decay to our signal modes, we fit $aM$(the average mass of $D_{s}$ at signal side and tag side), which is shown in Fig.\ref{DT-fit}(b). 
            Here, the background is described with $2^{nd}$-order Chebychev polynomial. The double tag yield is $1629\pm43$. 
    \end{itemize}

    \begin{figure*}[!htbp]
        \centering
        \includegraphics[width=0.4\textwidth]{plot/DT-A.eps}
        \includegraphics[width=0.4\textwidth]{plot/DT-B.eps}
        \caption{Fit of (a)Cat. A and (b)Cat. B.
            We fit $M_{sig}$ and $aM$ for Cat. A and Cat. B, respectively. The signal shapes are the corresponding simulated shapes convoluted with a Gaussian function and 
        the background shapes are described with $2^{nd}$-order Chebychev polynomial.}
        \label{DT-fit}
    \end{figure*}



    Then we can get the branching fraction $\mathcal{B}(D_{s}^{+} \rightarrow K^{+}K^{-}\pi^{+})=(5.38\pm0.07)\%$(only statistical uncertainty).

    \subsection{Systematic Uncertainties}
    The following sources are taken in account to calculate systematic uncertainties.

    \begin{itemize}
        \item Signal shape. The systematic uncertainty due to the signal shape is studied with the fit without the Gaussian function convoluted, the double tag yield shift is taken as the related effect. 

        \item Background shape and fit range. For background shape and the fit range, the MC shape is used to instead the $1^{st}$-order Chebychev polynomial and the fit range of $[1.90, 2.03]GeV$ is changed to $[1.89, 2.04]GeV$. 
            The largest branching fraction shift is taken as the related effect.
        
        \item Fit bias. The possible bias is estimated by the IO check with using the round 30-40 of generic MC.
            
        \item $K^{\pm}$ and $\pi^{\pm}$ Tracking/PID efficiency. Based on the works ~\cite{PID} and ~\cite{Tracking} by Xingyu Shan and Sanqiang Qu, etc, 
            we find that it's enough to assign $1.1\%$ , $0.4\%$, $1.1\%$ and $0.2\%$ as the systematic uncertainty for $K^{\pm}$ PID, $\pi^{\pm}$ PID,  $K^{\pm}$ tracking, $\pi^{\pm}$ tracking efficiencies, respectively.
    
    \item MC statistics. The uncertainty of MC statistics is obtained by $\sqrt{ \begin{matrix} \sum_{i} f_{i}\frac{\delta_{\epsilon_{i}}}{\epsilon_{i}}i\end{matrix}}$, where $f_{i}$ is the tag yield fraction and $\epsilon_{i}$ is the signal efficiency of tag mode $i$.
    \end{itemize}

    All of the systematic uncertainties mentioned above are summarized in Tab.\ref{BF-Sys}.
    \begin{table}
        \caption{Systematic uncertainties of branching fraction.}
        \label{BF-Sys}
        \begin{center}
            \begin{tabular}{cccc}
                \toprule
                Source   & Sys. Uncertainty\\
                \hline
                Signal shale                        & 0.3 \\
                Background shape and fit range      & 1.1 \\
                Fit bias                            & 0.2 \\
                $K^{\pm}$ PID efficiency            & 1.1 \\
                $\pi^{\pm}$ PID efficiency          & 0.4 \\
                $K^{\pm}$ Tracking efficiency       & 1.1 \\
                $\pi^{\pm}$ Tracking efficiency     & 0.2 \\
                MC statistics                       & 0.2 \\
                \hline
                total                               & 2.0 \\
                \bottomrule
            \end{tabular}
        \end{center}
    \end{table}

    The branching farction with systematic uncertainties is $\mathcal{B}(D_{s}^{+} \rightarrow K^{+}K^{-}\pi^{+})=(5.38\pm0.07_{stat.}\pm0.11_{sys.})\%$.




}
