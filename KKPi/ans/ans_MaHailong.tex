\documentclass[a4paper]{article}
\usepackage{color}
\usepackage{epsfig}
\begin{document}
\section{the reply to MaHaiLong's questions }
%@ Sun Aug 25 19:46:15 CST 2019
\begin{itemize}
    \item 1. Sec. 1.1.  You use long subsection to explain how difficult to distinguish f0(980) and a0(980), then you use combined amplitude to describe them in your amplitude analysis. 
        I am not very sure whether it is necessary to mention this in INTRODUCTION. 
        Should it be better to directly mention it in your amplitude analysis part?

        {\color{blue} 
            We prefer to keep the long discussion in INTRODUCTION on page 3-6.
            We explain the difficulty here to derive the definition of S(980), the shape of which is extracted in Sec. 4 and then used in amplitude analysis part in Sec. 5.
        }

    \item 2. Figure 22. The left sub-figure seems not very good. You can see that the data points of Ds peak are almost on the fitted curves.
        Have you checked whether smearing Gaussian improve the fit?

        {\color{blue}Figure 24 on page 37 has been updated.}


    \item 3. Line 16 on page 30.
        You use DIY MC to estimate detection efficiency based on your amplitude analysis results. 
        Have you examined whether there will be some uncertainty due to the fluctuations of your fitted results? If it is negligible, it is better to mention this in your systematic part.

        {\color{blue}See Sec. 6.4 on page 39.
            The systematic uncertainty of Dalitz model is added.
        }

    \item 4. Systematic uncertainty of BF measurement, or Table 14: 
        Can the systematic uncertainty of ST Ds yields can be really negligible? 
        Especially for the uncertainty due to background fluctuation?

        {\color{blue}See Sec. 6.4 on page 37. The systematic uncertainty of ST Ds yields due to the signal shapes, background shapes and fit ranges has been updated.}

    \item 5. Sec. 6, BF measurement part: You obtain the DT yield by 2D fit to $M_{inv}^{tag}$ vs. $M_{inv}^{sig}$.
          Why need to obtain ST yields and efficiencies in given mass windows ?
        
          {\color{blue} 
              See Sec. 6.2 on page 36 and 37.
              We do not do 2D fit to get ST yields.
              For Cat. B (Both the signal modes and the tag modes are $D_{s}^{+} \rightarrow K^{+}K^{-}\pi^{+}$.), we fit the average mass of Ds at signal side and tag side. 
              For tag mode $D_{s}^{-} \rightarrow K^{+}K^{-}\pi^{-}$, we do not constrain the mass window when obtaining the ST yields and efficiencies (on page 33 Table 14) .
          }

      \item 6. Table 11:Please mention the BFs of sub-decays are not included in effs.

          {\color{blue}See Sec. 6.1 on page 34. 
              Table 14 has been updated ( page 34).
          }

      \item 7. Have you checked whether your obtained BF of Ds $\rightarrow$ a0(980)pi agrees with the one obtained in the Ds $\rightarrow$ etapi+pi0 measurements?
            It is better to mention this situation in the summary part.

            {\color{blue} See Sec. 7 on page 40. 
                The comparisons of $\mathcal{B}(D_{s}^{+} \rightarrow S(980)\pi^{+})$ in this analysis and $\mathcal{B}(D_{s}^{+} \rightarrow a(980)\pi^{+})$ in the $D_{s}^{+} \rightarrow \pi^{+}\pi^{0}\eta$ measurements is updated in the summary part.
            }

    \item 8. Summary: In summary and future paper, it should be valuable to provide the BFs of sub-decays. 
        And then, it is better compare some VP decays, e.g. Ds $\rightarrow$ K*K and phipi, which can be predicted by theory.
        
        {\color{blue} See Sec. 7 on page 40.
            The BFs of sub-decays has been added in the summary part (page 36).
        }

    \item 9.  others. L8 on page 3: meson $\rightarrow$ mesons Table 12, the 2nd row from bottom: pi $\rightarrow$ $\pi$. 

        {\color{blue} Grammar errors mentioned has been modified.}
\end{itemize}
\end{document}
