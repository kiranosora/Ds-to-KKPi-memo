\documentclass[a4paper]{article}
\usepackage{color}
\usepackage{epsfig}
\begin{document}
\section{the reply to Jim Libby's questions}
%@Mon Sep 16 11:25:33 CST 2019
Re: Re: The MEMO for "Amplitude Analysis and Branching Fraction Measurement of $D_{s}^{+} \rightarrow K^{+}K^{-}\pi^{+}$ 
Sat Sep 21 13:08:16 CST 2019
\begin{itemize}
    \item 1. However, why not fit Fig 11 from the MIPWA with $|A(f0) + ae^{delta} A(a0)|^{2}$, where A(f0) and A(a0) are based on Eq. 11. This will give you a different parameterization to try in the amplitude fit. I am really not sure about this S(980) object as it is not physical.

        {\color{blue}
            As a0 and f0 is so close that I think it is not appropriate to describe them with the sum of them.
            I think here we should use K-matrix formalism (See page 74 of the attachment Physicsbes3.pdf). 
            The formalism used here is
            \begin{equation}
                \begin{array}{lr}
                    \hat{T} = (I - i\hat{K}\rho)^{-1} \hat{K}&\\
                    \hat{K} = \frac{m_{\alpha}\Gamma_{\alpha}}{m_{\alpha}^{2} - m_{\alpha}^{2}} + \frac{m_{\beta}\Gamma_{\beta}}{m_{\beta}^{2} - m_{\beta}^{2}}, &
                \end{array}\label{K-matrix} 
            \end{equation}

            Then fit the S wave shape with $|\hat{T}|^{2}$, the result is shown in Fig.~\ref{K-matrix-Fit}.

            \begin{figure*}[htbp]
                \centering
                \includegraphics[width=0.99\textwidth]{K-matrix-Fit.eps}
                \caption{Fit with $|\hat{T}|^{2}$.}
                \label{K-matrix-Fit}
            \end{figure*}
            The correlation coefficient between $m(\alpha)$ and $m(\beta)$ is 0.973, which is too large.
            It's very difficult to distinguish them as the mass of a0 and f0 are both under the K+K- threshold and a0 and f0 are very close.

        }


    \item 2. How the pull study results are turned into a systematic is still not explained.

        {\color{blue} 
            See Sec. 5.6 on page 30.
            The quadrature sum of the difference of the mean and the nominal value and the error of mean is considered as the uncertainty associated with fit bias.
        }

\end{itemize}
\end{document}
