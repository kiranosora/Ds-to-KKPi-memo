\section{Model Independent Partial Wave Analysis}
\par{In the $K^{+}K^{-}$ threshold both $a_{0}(980)$ and $f_{0}(980)$ can be present, and both resonances have very similar parameters which suffer from large uncertainties. In this section we obtain model-independent information on the $K^{+}K^{-} S$ wave by performing a MIPPPWA(model indendent partial wave analysis) in the  $K^{+}K^{-}$ threshold region.}
\subsection{Signal Candidate Selection}
\label{MIPWASelection}
\par{
    After the selection in Set.\ref{ST-selection}, we continue to select signals for model independent partial wave analysis.
    As MIPWA need more data events, we do not requre $D_{s}^{+}$ and $D_{s}^{-}$ appear in pairs. 
Before selecting the best candidate,  we vote the candidates with $\pi^{\pm}$($\pi^{0}$) whose momentum is less than $0.1GeV$ to remove soft $\pi^{\pm}$($\pi^{0}$) from $D^{*}$ decays.
For every candidates of $D_{s}$ decays, all tracks at signal side are added to apply kinematric fitting. Only mass of $D_{s}$ from signal side is constrained.Then we select the candidate with minimum $\chi_{1c}^{2}$.
}
\subsection{Background Analysis}
In order to further suppress the background, the multiple-variable analysis (MVA) is used. We train MVA methos separately with different sets of variables for the two event categories depending on the $D_{s}^{+}$ origin. 
Sideband region used below is defined as the region of  $1.90 < M(D_{s}) < 1.952 GeV$ and   $1.985 < M(D_{s}) < 2.03 GeV$, $M(D_{s})$ is the invariant mass of $D_{s}$.
And the signal region is $1.952 < M(D_{s}) < 1.985 GeV$.
These two categories of events are selected in a $M_{rec}-\Delta{M}$ 2D plane shown in Fig.\ref{2DAll}:

\begin{figure*}[h]
 \centering
 \mbox{
  %\vskip -1.5cm
  \begin{overpic}[width=0.48\textwidth]{plot/2DAll.png}
  \end{overpic}
 }
 \caption{Two dimensional plane of ${M_{rec}}$ versus ${\Delta{M} \equiv M(D_{s}^{+}\gamma) - M(D_{s}^{+})}$ from the simulated ${D_{s}^{+} \rightarrow K^{+}K^{-}\pi^{+}}$ decyas. The red(green) dashed lines mark the mass window for the $D_{s}^{+}$ Cat. \#0( Cat. \#1) around the $M_{rec}$($\Delta{M}$) peak. }
\label{2DAll}
\end{figure*}



\begin{itemize}
    \item Cat. \#0: Direct $D_{s}^{+}$. We use the following variables whose distributions for signal and background are shown in Fig.\ref{Cat0Variables},
        \begin{itemize}
            \item[1. ] $M_{rec}$,
            \item[2. ] $P_{rest}$, defined as the total momentum of the tracks and neutrals in the rest of event (not part of the $D_{s}^{+} \rightarrow K^{+}K^{-}\pi^{+}$ candidate),
            \item[3. ] $E_{\gamma}$, defined as the energy of gamma from $D_{s}^{*}$.
        \end{itemize}
        From Fig. \ref{Cat0Variables}(generic MC) and Fig.\ref{cat0_data}(data), we can see that the corresponding distributions of these variables of data and generic MC are roughly consistent.
        

    \item Cat. \#1: Indirect $D_{s}^{+}$.
        We use the following variables whose distributions for signal and background are shown in Fig.\ref{Cat1Variables},
        \begin{itemize}
            \item[1. ] $\Delta{M}$,
            \item[2. ] $M_{rec}^{'}$, defined as $M_{rec}^{'} = \sqrt{ {(E_{cm}  - \sqrt{p_{D_{s}\gamma}^{2} + m_{D_{s}^{*}}^{2}}) }^{2} - p_{D_{s}\gamma}^{2}}$, with $p_{D_{s}\gamma}$ as the momentum of the $D_{s}\gamma$ combination, $m_{D_{s}^{*}}$ as the nominal ${D_{s}^{*}}$ mass,
            \item[3. ] $N_{tracks}$, defined as the total number of tracks and neutrals in an event.
        \end{itemize}
        From Fig. \ref{Cat1Variables}(generic MC) and Fig.\ref{cat1_data}(data), we can see that the corresponding distributions of these variables of data and generic MC are roughly consistent.
\end{itemize}


\begin{figure*}[h]
 \centering
 \mbox{
  %\vskip -1.5cm
  \begin{overpic}[width=0.99\textwidth]{plot/Cat0Variables.png}
  \end{overpic}
 }
 \caption{ For event Cat. \#0,  distributions of MVA variables from simulated signal decays and background events.}
\label{Cat0Variables}
\end{figure*}

\begin{figure*}[!htbp]
 \centering
 \includegraphics[width=0.99\textwidth]{plot/cat0_sig.eps}
 \includegraphics[width=0.99\textwidth]{plot/cat0_sb.eps}
 \caption{The distribution(Cat. \#0) of these three observables ((a) and (d)) $M_{rec}$, ((b) and (e)) $P_{rest}$ and  ((c) and (f)) $E_{\gamma}$ for( (a), (b) and (c)) Signal and ( (d), (e) and (f)) Sideband regions from data are shown.}
\label{cat0_data}
\end{figure*}

\begin{figure*}[h]
 \centering
 \mbox{
  %\vskip -1.5cm
  \begin{overpic}[width=0.99\textwidth]{plot/Cat1Variables.png}
  \end{overpic}
 }
 \caption{ For event Cat. \#1,  distributions of MVA variables from simulated signal decays and background events.}
\label{Cat1Variables}
\end{figure*}

\begin{figure*}[!htbp]
 \centering
 \includegraphics[width=0.99\textwidth]{plot/cat1_sig.eps}
 \includegraphics[width=0.99\textwidth]{plot/cat1_sb.eps}
 \caption{The distribution(Cat. \#1) of these three observables ((a) and (d)) $M_{rec}$, ((b) and (e)) $P_{rest}$ and  ((c) and (f)) $E_{\gamma}$ for( (a), (b) and (c)) Signal and ( (d), (e) and (f)) Sideband regions from data are shown.}
\label{cat1_data}
\end{figure*}

\par{As the results shown in Fig.\ref{Overtrain}, this BDTG training and test samples are well matched. For event Cat. \#0 ( Cat. \#1 ), the sample with BDTG value larger than 0.33(0.65) is retained for further study. 
    
\begin{figure*}[h]
    \centering
    \mbox{
        %\vskip -1.5cm
        \begin{overpic}[width=0.48\textwidth]{plot/Cat0Overtrain.eps}
        \end{overpic}
    }
    \mbox{
        %\vskip -1.5cm
        \begin{overpic}[width=0.48\textwidth]{plot/Cat1Overtrain.eps}
        \end{overpic}
    }
    \caption{ The comparisions between the training and test samples.}
    \label{Overtrain}
\end{figure*}
    
   
    After applying the BDTG requirement, the background shows no obviously peak around the region of [1.95, 1.982] $GeV$ (Signal region), which are shown in Fig.\ref{MIPWA-ST-BS}.  The fit to the signal $D_{s}$ invariant mass ($M_{D_{s}^{2}}$) spectrum gives the background yield in Signal region is $73.6\pm18.7$, shown as in Fig.\ref{MIPWA-ST}. In the fit, the signal shape is the MC shape convoluted with a Gaussian function and the background is described with $1^{st}$-order Chebychev polynomial.

    \begin{figure*}[h]
        \centering
        \mbox{
            %\vskip -1.5cm
            \begin{overpic}[width=0.48\textwidth]{plot/MIPWA-sig.eps}
            \end{overpic}
        }
        \mbox{
            %\vskip -1.5cm
            \begin{overpic}[width=0.48\textwidth]{plot/MIPWA-bkg.eps}
            \end{overpic}
        }
        \caption{ The signal and background distributions from generic MC after BDTG requirement.}
        \label{MIPWA-ST-BS}
    \end{figure*}

}

\begin{figure*}[h]
 \centering
 \mbox{
  %\vskip -1.5cm
  \begin{overpic}[width=0.8\textwidth]{plot/MIPWA-ST.png}
  \end{overpic}
 }
 \caption{The fit to the signal $D_{s}$ invariant mass ($M_{D_{s}}$) spectrum after BDTG requirement, the area between the pink arrows is the signal area of the sample for MIPWA. }
\label{MIPWA-ST}
\end{figure*}

\par{
The projections of the "Sideband"( $1.90 < M(D_{s}) < 1.95 GeV$ and   $1.985 < M(D_{s}) < 2.03 GeV$) from data and generic MC after signal events removed are shown in Fig.\ref{MIPWA-Sideband}. The corresponding plots agree well.} 



\begin{figure*}[h]
    \centering
    \mbox{
        %\vskip -1.5cm
        \begin{overpic}[width=0.3\textwidth]{plot/MIPWA-m12.png}
        \end{overpic}
    }
    \mbox{
        %\vskip -1.5cm
        \begin{overpic}[width=0.3\textwidth]{plot/MIPWA-m13.png}
        \end{overpic}
    }
    \mbox{
        %\vskip -1.5cm
        \begin{overpic}[width=0.3\textwidth]{plot/MIPWA-m23.png}
        \end{overpic}
    }
    \caption{ The projections of $m(K^{+}K^{-})$, $m(K^{-}\pi^{+})$, $m(K^{+}\pi^{+})$ from "Sideband" for data(dots with error bars) and generic MC after signal events removed (red histogram) after BDTG requirement.}
    \label{MIPWA-Sideband}
\end{figure*}

\par{Thus the generic MC sample with signal events removed is used to subtract the background in MIPWA.
}
































\subsection{MIPWA method}
\par{Let N be the number of events for a given mass interval $I=[m_{K^{+}K^{-}}; m_{K^{+}K^{-}} + dm_{K^{+}K^{-}}]$. We write the corresponding angular distributions in terms of the appropriate spherical harmonic functions as
    \begin{equation}
        \frac{dN}{d\cos\theta} = 2\pi\sum_{k=0}^L\left\langle Y_{k}^{0}\right\rangle Y_{k}^{0}(\cos\theta),\label{expansion}
    \end{equation}
    where $L - 2 \ell_{max}$, and $\ell_{max}$ is the maximum orbital angular momentum quantum number required to describe the $K^{+}K^{-}$ system at $m_{K^{+}K^{-}}$ (e.g. $\ell_{max}$ =1 for S-, P-wave description); $\theta$ is the angle between the $K^{+}$ direction in the $K^{+}K^{-}$ system in the $D_{s}^{+}$ rest frame. The normalizations are such that
    \begin{equation}
        \int_{-1}^{1}Y_{k}^{0}(\cos\theta)Y_{j}^{0}(\cos\theta), d\cos\theta  = \frac{\delta_{kj}}{2\pi},\label{sh-normalizations}
    \end{equation}
    and it is  assumed that the distribution $\frac{dN}{d\cos\theta}$ has been efficiency corrected and background subtracted.
    Using this orthogonality condition, the coefficiencies in the expansion are obtained from 
    \begin{equation}
        \left\langle Y_{k}^{0} \right\rangle = \int_{-1}^{1} \frac{dN}{d\cos\theta} d\cos\theta,\label{expansion-coefficiencies}
    \end{equation}
    where the intergral is given, to a good approximiation, by $\sum_{n=1}^{N}Y_{k}^{0}(\cos\theta_{n})$, where $\theta_{n}$ is the value of $\theta$ for the $n$-th event.
    
    Fig.\ref{Y0} shows the $K^{+}K^{-}$ mass spectrum up to 1.15 GeV weighted by $Y_{k}^{0}(\cos\theta) = \sqrt{(2k+1)/(4\pi)}P_{k}(\cos\theta)$ for k=0, 1, and 2, where $P_{k}$ is the Legendre ploynomial function of orkder $k$. These distributions are corrected for efficiency and phase space, and background is subtracted using background from generic MC after BDTG requirement.
    
    The number of events N for the mass interval $I$ can be expressed also in terms of the partial-wave amplitudes describing the $K^{+}K^{-}$ system. Assuming that only S- and P-wave amplitudes are necessary in this limited region, we can write:
    \begin{equation}
        \frac{dN}{d\cos\theta} = 2\pi\left|SY_{0}^{0}(\cos\theta) + PY_{1}^{0}(\cos\theta)\right|^{2}.\label{SP-distribution}
    \end{equation}
    By comparing Eqa. \ref{expansion} and \ref{SP-distribution}, we obtain 
    \begin{equation}
        \begin{array}{lr}
            \sqrt{4\pi}\left\langle Y_{0}^{0}\right\rangle = \left|S\right|^{2} + \left|P\right|^{2}, &\\ 
            \sqrt{4\pi}\left\langle Y_{2}^{0}\right\rangle = \frac{2}{\sqrt{5}}\left|P\right|^{2}, &
        \end{array}\label{SP-RES} 
    \end{equation}

}

\begin{figure*}[h]
    \centering
    \mbox{
  %\vskip -1.5cm
  \begin{overpic}[width=0.8\textwidth]{plot/Y0.png}
  \end{overpic}
 }
 \caption{ $K^{+}K^{-}$ mass spectrum in the threshold region weighted by (a) $Y_{0}^{0}$, (b) $Y_{1}^{0}$ and (c) $Y_{2}^{0}$, corrected for efficiency and phase space, and background subtracted. }
\label{Y0}
\end{figure*}

\par{The above system of equations can be solved in each interval of $K^{+}K^{-}$ invariant mass for $\left|S\right|$ and $\left|P\right|$ and the resulting distributions are shown in Fig.\ref{SP}.  
}

\begin{figure*}[h]
    \centering
    \mbox{
  %\vskip -1.5cm
  \begin{overpic}[width=0.8\textwidth]{plot/SP.png}
  \end{overpic}
 }
 \caption{ Squared (a) S- and (b) P-wave amplitudes}
 \label{SP}
\end{figure*}

\subsection{S-wave parameterization at the $K^{+}K^{-}$ threshold}
\par{We empirically parameterize the $f_{0}(980)$ with the following function:
    \begin{equation}
        A_{f_{0}(980) / a_{0}(980)} = \frac{1}{m_{0}^{2} - m^{2} -im_{0}\Gamma_{0}\rho_{KK}}, \label{a0980-RBW}
    \end{equation}
    where $\rho_{KK} = 2p/m$, and obtain the flollowing parameter values:
    \begin{equation}
        \begin{array}{lr}
            m_{0} = (0.919 \pm 0.006_{stat}) GeV, &\\
            \Gamma_{0} = (0.272 \pm 0.040_{stat}) GeV. &
        \end{array}\label{S-wave parameters} 
    \end{equation}


    The errors are statistical only. The fit result are shown in Fig.\ref{FitSWave}.
    
    \begin{figure*}[h]
        \centering
        \mbox{
            %\vskip -1.5cm
            \begin{overpic}[width=0.8\textwidth]{plot/Fit-SWave.eps}
            \end{overpic}
        }
        \caption{ Fit of squared S-wave amplitudes. the curves result from the fit described in the text.}
        \label{FitSWave}
    \end{figure*}
}

