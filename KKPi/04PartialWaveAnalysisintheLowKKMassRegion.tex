\section{Partial Wave Analysis in the Low $K^{+}K^{-}$ Mass Region}
\label{MIPWA}
\par{In the $K^{+}K^{-}$ threshold, both $a_{0}(980)$ and $f_{0}(980)$ can be present, and both resonances have very similar parameters which suffer from large uncertainties. 
In this section we obtain the model-independent information on the $K^{+}K^{-} S$ wave by performing a partial wave analysis (PWA) in the  $K^{+}K^{-}$ threshold region.}
\subsection{Event Selection}
\label{MIPWASelection}
\par{
    After the selection in Sec.~\ref{ST-selection}, we select signals for the model independent partial wave analysis.
    To retain enough statistics, we decide to use the single-tag method, and only to fully reconstruct one $D_{s}^{+} \rightarrow K^{+}K^{-}\pi^{+}$ in each event. 
We veto the candidates with $\pi^{\pm}$ ($\pi^{0}$) whose momentum is less than $0.1$ GeV to remove soft $\pi^{\pm}$ ($\pi^{0}$) from $D^{*}$ decays.
For each $D_{s}$ candidate, all daughter tracks are added to apply a 1C kinematic fit constraining the mass of $D_{s}$. 
Then we select the best candidate with minimum $\chi_{1c}^{2}$.
}
\subsection{Background Analysis}
\label{MIPWA-BA}
In order to suppress the background, the multiple-variable analysis (MVA) is used. We train MVA separately with different sets of variables for the two event categories depending on the $D_{s}^{+}$ origin. 
Sideband region used below is defined as the region of  $1.90 < M(D_{s}) < 1.95$ GeV/$c^{2}$ and   $1.986 < M(D_{s}) < 2.03$ GeV/$c^{2}$, where $M(D_{s})$ is the invariant mass of $D_{s}$ before applying the kinematic fit, and the signal region is $1.95 < M(D_{s}) < 1.986$ GeV/$c^{2}$.
Two categories of events are selected in a $M_{rec}-\Delta{M}$ 2D plane as shown in Fig.~\ref{2DAll}.

\begin{figure*}[htbp]
 \centering
 \mbox{
  %\vskip -1.5cm
  \begin{overpic}[width=0.48\textwidth]{plot/2DAll.png}
  \end{overpic}
 }
 \caption{Two dimensional plane of ${M_{rec}}$ versus ${\Delta{M} \equiv M(D_{s}^{+}\gamma) - M(D_{s}^{+})}$ from the simulated ${D_{s}^{+} \rightarrow K^{+}K^{-}\pi^{+}}$ decays. The red (green) dashed lines mark the mass window for the $D_{s}^{+}$ Cat.~\#0 ( Cat.~\#1) around the $M_{rec}$ ($\Delta{M}$) peak. }
\label{2DAll}
\end{figure*}



\begin{itemize}
    \item Cat. \#0: Direct $D_{s}^{+}$. We use the following variables whose distributions for signal and background are shown in Fig.~\ref{Cat0Variables} for generic MC and Fig.~\ref{cat0_data} for data,
        \begin{itemize}
            \item[1. ] $M_{rec}$,
            \item[2. ] $P_{rest}$, defined as the total momentum of the tracks and neutrals in the rest of event (not part of the $D_{s}^{+} \rightarrow K^{+}K^{-}\pi^{+}$ candidate),
            \item[3. ] $E_{\gamma}$, defined as the energy of gamma from $D_{s}^{*}$.
        \end{itemize}
        From Fig.~\ref{Cat0Variables} and Fig.~\ref{cat0_data}, we can see that the corresponding distributions of these variables of data and generic MC are roughly consistent.
        

    \item Cat. \#1: Indirect $D_{s}^{+}$.
        We use the following variables whose distributions for signal and background are shown in Fig.~\ref{Cat1Variables} for generic MC and Fig.~\ref{cat1_data} for data,
        \begin{itemize}
            \item[1. ] $\Delta{M}$,
            \item[2. ] $M_{rec}^{'}$, defined as $M_{rec}^{'} = \sqrt{ {(E_{cm}  - \sqrt{p_{D_{s}\gamma}^{2} + m_{D_{s}^{*}}^{2}}) }^{2} - p_{D_{s}\gamma}^{2}}$, with $p_{D_{s}\gamma}$ as the momentum of the $D_{s}\gamma$ combination, $m_{D_{s}^{*}}$ as the nominal ${D_{s}^{*}}$ mass,
            \item[3. ] $N_{tracks}$, defined as the total number of tracks and neutrals in an event.
        \end{itemize}
        From Fig.~\ref{Cat1Variables} and Fig.~\ref{cat1_data}, we can see that the corresponding distributions of these variables of data and generic MC are roughly consistent.
\end{itemize}


\begin{figure*}[htbp]
 \centering
 \mbox{
  %\vskip -1.5cm
  \begin{overpic}[width=0.99\textwidth]{plot/Cat0Variables.png}
  \end{overpic}
 }
 \caption{ For event Cat. \#0,  distributions of MVA variables from simulated signal decays and background events.}
\label{Cat0Variables}
\end{figure*}

\begin{figure*}[htbp]
 \centering
 \includegraphics[width=0.99\textwidth]{plot/cat0_sig.eps}
 \includegraphics[width=0.99\textwidth]{plot/cat0_sb.eps}
 \caption{The distributions (Cat. \#0) of these three observables ((a) and (d)) $M_{rec}$, ((b) and (e)) $P_{rest}$ and  ((c) and (f)) $E_{\gamma}$ for( (a), (b) and (c)) signal and ( (d), (e) and (f)) sideband regions of data.}
\label{cat0_data}
\end{figure*}

\begin{figure*}[htbp]
 \centering
 \mbox{
  %\vskip -1.5cm
  \begin{overpic}[width=0.99\textwidth]{plot/Cat1Variables.png}
  \end{overpic}
 }
 \caption{ For event Cat. \#1,  distributions of MVA variables from simulated signal decays and background events.}
\label{Cat1Variables}
\end{figure*}

\begin{figure*}[htbp]
 \centering
 \includegraphics[width=0.99\textwidth]{plot/cat1_sig.eps}
 \includegraphics[width=0.99\textwidth]{plot/cat1_sb.eps}
 \caption{The distributions (Cat. \#1) of these three observables ((a) and (d)) $\Delta M$, ((b) and (e)) $M_{rec}^{'}$ and  ((c) and (f)) $N_{tracks}$ for( (a), (b) and (c)) signal and ( (d), (e) and (f)) sideband regions from data.}
\label{cat1_data}
\end{figure*}

\par{As the results shown in Fig.~\ref{Overtrain}, BDTG training and test samples are well matched. For event Cat. \#0 ( Cat. \#1 ), the sample with BDTG value larger than 0.33 (0.65) is retained for further study.
    With the BDTG cut criteria, we can get a relatively pure sample (background less than 4\%) and the background ratios of Cat. \#0 and Cat. \#1 are almost the same.
    
\begin{figure*}[htbp]
    \centering
    \mbox{
        %\vskip -1.5cm
        \begin{overpic}[width=0.48\textwidth]{plot/Cat0Overtrain.eps}
        \end{overpic}
    }
    \mbox{
        %\vskip -1.5cm
        \begin{overpic}[width=0.48\textwidth]{plot/Cat1Overtrain.eps}
        \end{overpic}
    }
    \caption{ The comparisons between the training and test samples. The plot at left (right) is the comparision of Cat. \#0 (Cat. \#1).}
    \label{Overtrain}
\end{figure*}
    
   
    After applying the BDTG requirement, the background shows no obviously peak around the region of [1.95, 1.986] GeV/$c^{2}$ (Signal region), which are shown in Fig.~\ref{MIPWA-ST-BS}.  
    The fit to the signal $D_{s}$ invariant mass gives the background yield in Signal region is $735.7\pm30.0$, as shown in Fig.~\ref{MIPWA-ST}.
    In the fit, the signal shape is the MC shape convoluted with a Gaussian function and the background is described with second-order Chebychev polynomial.

    \begin{figure*}[htbp]
        \centering
        \mbox{
            %\vskip -1.5cm
            \begin{overpic}[width=0.48\textwidth]{plot/MIPWA-sig.eps}
            \end{overpic}
        }
        \mbox{
            %\vskip -1.5cm
            \begin{overpic}[width=0.48\textwidth]{plot/MIPWA-bkg.eps}
            \end{overpic}
        }
        \caption{ The signal and background distributions from generic MC after BDTG requirement.}
        \label{MIPWA-ST-BS}
    \end{figure*}

}

\begin{figure*}[htbp]
 \centering
 \mbox{
  %\vskip -1.5cm
  \begin{overpic}[width=0.48\textwidth]{plot/MIPWA-ST.eps}
  \end{overpic}
 }
 \caption{The fit to the signal $D_{s}$ invariant mass ($M_{D_{s}}$) spectrum (the dots with error bars) after BDTG requirement, the area between the pink arrows is the signal area of the sample for MIPWA.
     The signal shape (green line) is the MC shape convoluted with a Gaussian function and the background shape (red line) is second-order Chebychev polynomial.
 }
\label{MIPWA-ST}
\end{figure*}

\par{
The projections of the ``sideband" ( $1.90 < M(D_{s}) < 1.95$ GeV/$c^{2}$ and   $1.986 < M(D_{s}) < 2.03$ GeV/$c^{2}$) from data and generic MC with signal events removed are shown in Fig.~\ref{MIPWA-Sideband}. 
The corresponding plots agree well.
Thus the generic MC sample with signal events removed is used to subtract the background in data.
} 



\begin{figure*}[htbp]
    \centering
    \mbox{
        %\vskip -1.5cm
        \begin{overpic}[width=0.3\textwidth]{plot/MIPWA-m12.eps}
        \end{overpic}
    }
    \mbox{
        %\vskip -1.5cm
        \begin{overpic}[width=0.3\textwidth]{plot/MIPWA-m13.eps}
        \end{overpic}
    }
    \mbox{
        %\vskip -1.5cm
        \begin{overpic}[width=0.3\textwidth]{plot/MIPWA-m23.eps}
        \end{overpic}
    }
    \caption{ The Dalitz projections to $m(K^{+}K^{-})$ ( (a) ), $m(K^{-}\pi^{+})$ ( (b) ), $m(K^{+}\pi^{+})$ ( (c) ) from ``Sideband" for data (dots with error bars) and generic MC with signal events removed (red histogram) after BDTG requirement.}
    \label{MIPWA-Sideband}
\end{figure*}

\par{
    
    Figure~\ref{BDTG-IO} shows the shape comparison between data ( (a), (b) and (c) ) and MC ( (d), (e) and (f) ) for the inputs and outputs of BDTG.
    We can see that there is no significant difference.
    The background has been subtracted using the corresponding shape of ``Sideband" in this shape comparison.
    \begin{figure*}[!htbp]
        \centering
        \includegraphics[width=0.3\textwidth]{plot/MIPWA-Data-m12.eps}
        \includegraphics[width=0.3\textwidth]{plot/MIPWA-Data-m13.eps}
        \includegraphics[width=0.3\textwidth]{plot/MIPWA-Data-m23.eps}
        \includegraphics[width=0.3\textwidth]{plot/MIPWA-MC-m12.eps}
        \includegraphics[width=0.3\textwidth]{plot/MIPWA-MC-m13.eps}
        \includegraphics[width=0.3\textwidth]{plot/MIPWA-MC-m23.eps}
        \caption{The shape comparison between data ( (a), (b) and (c) ) and MC ( (d), (e) and (f) ) for the inputs and outputs of BDTG. 
        The background has been subtracted.}
        \label{BDTG-IO}
    \end{figure*}

}


\subsection{Partial Wave Analysis}
\par{Assuming N is the number of events for a given mass interval $I=[m_{K^{+}K^{-}}; m_{K^{+}K^{-}} + dm_{K^{+}K^{-}}]$, we write the corresponding angular distributions in terms of the appropriate spherical harmonic functions as 
    \begin{equation}
        \frac{dN}{d\cos\theta} = 2\pi\sum_{k=0}^L\left\langle Y_{k}^{0}\right\rangle Y_{k}^{0}(\cos\theta),\label{expansion}
    \end{equation}
    where $L = 2 \ell_{max}$, and $\ell_{max}$ is the maximum orbital angular momentum quantum number required to describe the $K^{+}K^{-}$ system at $m_{K^{+}K^{-}}$ (e.g. $\ell_{max}$ =1 for S-, P-wave description); $\theta$ is the angle between the $K^{+}$ direction in the $K^{+}K^{-}$ rest frame and the prior direction of the $K^{+}K^{-}$ system in the $D_{s}^{+}$ rest frame. 
    The normalizations are 
    \begin{equation}
        \int_{-1}^{1}Y_{k}^{0}(\cos\theta)Y_{j}^{0}(\cos\theta) d\cos\theta  = \frac{\delta_{kj}}{2\pi},\label{sh-normalizations}
    \end{equation}
    and it is  assumed that the distribution $\frac{dN}{d\cos\theta}$ has been efficiency corrected and background subtracted.
    Using this orthogonality condition, the coefficients in the expansion are obtained from 
    \begin{equation}
        \left\langle Y_{k}^{0} \right\rangle = \int_{-1}^{1}Y_{k}^{0}(\cos\theta) \frac{dN}{d\cos\theta} d\cos\theta. \label{expansion-coefficiencies}
    \end{equation}
    This integral is given, to a good approximation, by $\sum_{n=1}^{N}Y_{k}^{0}(\cos\theta_{n})$, where $\theta_{n}$ is the value of $\theta$ for the $n$-th event.
    
    Fig.~\ref{Y0} shows the $K^{+}K^{-}$ mass spectrum up to 1.15 GeV/$c^{2}$ weighted by $Y_{k}^{0}(\cos\theta) = \sqrt{(2k+1)/(4\pi)}P_{k}(\cos\theta)$ for k=0, 1, and 2, where $P_{k}$ is the Legendre polynomial function of order $k$. 
    These distributions are corrected for efficiency and phase space, and background is subtracted by background from generic MC after BDTG requirement.
    Before the efficiency correction, we subtract the background with the shape on the distribution $m_{K^{-}\pi^{+}}$ versus $m_{K^{+}K^{-}}$ from MC. 
    The distribution $m_{K^{-}\pi^{+}}$ versus $m_{K^{+}K^{-}}$ of MC is used to calculate the efficiency.
    Then we correct the distributions with the efficiency and the phase space factor $1/\sqrt{ 1 - \frac{4m(K)^{2}}{m(K^{+}K^{-})^{2}}}$, where $m(K)$ is the nominal mass of $K^{+}$ on PDG~\cite{PDG2018}.
    
    The number of events N for the mass interval $I$ can be expressed in terms of the partial-wave amplitudes describing the $K^{+}K^{-}$ system. Assuming that only S- and P-wave amplitudes are necessary in this limited region, we can write:
    \begin{equation}
        \frac{dN}{d\cos\theta} = 2\pi\left|SY_{0}^{0}(\cos\theta) + PY_{1}^{0}(\cos\theta)\right|^{2}.\label{SP-distribution}
    \end{equation}
    By comparing Eq.~\ref{expansion} and~\ref{SP-distribution}~\cite{PRD56-7299}, we obtain 
    \begin{equation}
        \begin{array}{lr}
            \sqrt{4\pi}\left\langle Y_{0}^{0}\right\rangle = \left|S\right|^{2} + \left|P\right|^{2}, &\\ 
            \sqrt{4\pi}\left\langle Y_{1}^{0}\right\rangle = 2\left|S\right|\left|P\right|cos\phi_{SP}, &\\ 
            \sqrt{4\pi}\left\langle Y_{2}^{0}\right\rangle = \frac{2}{\sqrt{5}}\left|P\right|^{2}, &
        \end{array}\label{SP-RES} 
    \end{equation}
    where $\phi_{SP} = \phi_{S} - \phi_{P}$ is the phase difference between S-wave and P-wave.


\begin{figure*}[htbp]
    \centering
    \mbox{
  %\vskip -1.5cm
  \begin{overpic}[width=0.98\textwidth]{plot/Y0.eps}
  \end{overpic}
 }
 \caption{ $K^{+}K^{-}$ mass spectrum in the threshold region weighted by (a) $Y_{0}^{0}$, (b) $Y_{1}^{0}$ and (c) $Y_{2}^{0}$, corrected for efficiency and phase space, and background subtracted. }
\label{Y0}
\end{figure*}

    The above system of equations can be solved in each interval of $K^{+}K^{-}$ invariant mass for $\left|S\right|$, $\left|P\right|$ and $\phi_{SP}$ and the resulting distributions are shown in Fig.~\ref{SP}. 
    In Fig.~\ref{SP}(c), we draw the phase difference twice because of the sign ambiguity associated with the value $\phi_{SP}$ extracted from cos$\phi_{SP}$.
    In the following section, we will discuss $\phi_{S}$ in Fig.~\ref{SP}(d).

}

\begin{figure*}[htbp]
    \centering
    \mbox{
  %\vskip -1.5cm
  \begin{overpic}[width=0.9\textwidth]{plot/SP.eps}
  \end{overpic}
 }
 \caption{ Squared (a) S- and (b) P-wave amplitudes; (c) the phase difference $\phi_{SP}$; (d) $\phi_{S}$ obtained as explained in the following section.}
 \label{SP}
\end{figure*}

\subsection{S-wave Parameterization at the $K^{+}K^{-}$ Threshold}
\label{MIPWA-RES}
\par{We empirically parameterize the $S(980)$ with the following function:
    \begin{equation}
        A_{S(980)} = \frac{1}{m_{0}^{2} - m^{2} -im_{0}\Gamma_{0}\rho_{KK}}, \label{S980-RBW}
        %A_{f_{0}(980) / a_{0}(980)} = \frac{1}{m_{0}^{2} - m^{2} -im_{0}\Gamma_{0}\rho_{KK}}, \label{S980-RBW}
    \end{equation}
    where $\rho_{KK} = 2p/m$, and obtain the following parameter values by fitting the S-wave amplitudes in Fig.~\ref{SP} (a) with this Eq.~\ref{S980-RBW}.:
    \begin{equation}
        \begin{array}{lr}
            m_{0} = (0.919 \pm 0.006_{stat}) \ {\rm GeV}/c^{2}, &\\
            \Gamma_{0} = (0.272 \pm 0.040_{stat}) \ {\rm GeV}. &
        \end{array}\label{S-wave parameters} 
    \end{equation}


    The errors are statistical only. The fit result is shown in Fig.~\ref{FitSWave}.
    
    \begin{figure*}[htbp]
        \centering
        \mbox{
            %\vskip -1.5cm
            \begin{overpic}[width=0.48\textwidth]{plot/Fit-SWave.eps}
            \end{overpic}
        }
        \caption{ Fit of S-wave amplitudes. The curves result from the fit described in the text.}
        \label{FitSWave}
    \end{figure*}
    From Eq.~\ref{RBW} in Sec.~\ref{propagator}, we can get the phase of $\phi(1020)$ $\phi_{P}$.
    Then we can obtain the phase of $S(980)$ $\phi_{S}$ by adding $\phi_{SP}$ and $\phi_{P}$.
    According to Eq.~\ref{S-wave parameters}, the $S(980)$ central mass is far away from the $K^{+}K^{-}$ threshold and $\phi_{S}$ should move slowly, so we choose the $\phi_{SP}$ which decreases rapidly near the $\phi(1020)$ peak as the physical solution.
    Then we can obtain the phase of $S(980)$ $\phi_{S}$, which is shown in Fig.\ref{SP}(d), by adding $\phi_{SP}$ and $\phi_{P}$.
The values of $\left|S\right|^{2}$ (arbitrary units), $\left|P\right|^{2}$ (arbitrary units) and $\phi_{S}$ are reported in Table~\ref{SP-values}. 
    \begin{table}[htbp]
        \caption{
            The values of $\left|S\right|^{2}$ (arbitrary units), $\left|P\right|^{2}$ (arbitrary units) and $\phi_{S}$.
            Some values of $\phi_{S}$ are missing because the corresponding $\left|P\right|^{2}$ is less than 0 and Eq.~\ref{SP-RES} cannot be solved. 
            Uncertainties in the table are statistical only.
        }
        \label{SP-values}
        \begin{center}
            \begin{tabular}{cccc}
                \toprule\toprule
                $m(K^{+}K^{-})$ mass interval (GeV/$c^{2}$) & $\left|S\right|^{2}$ (arbitrary units) & $\left|P\right|^{2}$ (arbitrary units) & $\phi_{S}$ (degrees)\\
                \hline
                %$[0.988, 0.992]$  &   14593$\ \pm\ $1860& -1137$\ \pm\ $1410  &   - \\ 
                $[0.988,\ 0.992]$   &	14593$\ \pm\ $1860&	-1137$\ \pm\ $1401&	 - \\ 	
                $[0.992,\ 0.996]$   &	11326$\ \pm\ $1364&	168$\ \pm\ $1027&	92$\ \pm\ $48 \\ 	
                $[0.996,\ 1.000]$   &	11064$\ \pm\ $1143&	-531$\ \pm\ $850&	 - \\ 	
                $[1.000,\ 1.004]$   &	8659$\ \pm\ $1015&	1006$\ \pm\ $748&	90$\ \pm\ $7 \\ 	
                $[1.004,\ 1.008]$   &	7207$\ \pm\ $1281&	7292$\ \pm\ $1003&	80$\ \pm\ $5 \\ 	
                $[1.008,\ 1.012]$   &	8703$\ \pm\ $1509&	11746$\ \pm\ $1200&	81$\ \pm\ $5 \\ 	
                $[1.012,\ 1.016]$   &	6669$\ \pm\ $2565&	48763$\ \pm\ $2066&	79$\ \pm\ $8 \\ 	
                $[1.016,\ 1.020]$   &	7051$\ \pm\ $6057&	199740$\ \pm\ $5048&	101$\ \pm\ $32 \\ 	
                $[1.020,\ 1.024]$   &	2466$\ \pm\ $4232&	122645$\ \pm\ $3520&	96$\ \pm\ $52 \\ 	
                $[1.024,\ 1.028]$   &	4292$\ \pm\ $2108&	34363$\ \pm\ $1748&	87$\ \pm\ $14 \\ 	
                $[1.028,\ 1.033]$   &	4009$\ \pm\ $1455&	15046$\ \pm\ $1212&	81$\ \pm\ $13 \\ 	
                $[1.033,\ 1.037]$   &	3922$\ \pm\ $1088&	8108$\ \pm\ $887&	78$\ \pm\ $14 \\ 	
                $[1.037,\ 1.041]$   &	3480$\ \pm\ $944&	5945$\ \pm\ $768&	70$\ \pm\ $14 \\ 	
                $[1.041,\ 1.045]$   &	5376$\ \pm\ $854&	3707$\ \pm\ $678&	71$\ \pm\ $14 \\ 	
                $[1.045,\ 1.049]$   &	4043$\ \pm\ $696&	2103$\ \pm\ $551&	76$\ \pm\ $14 \\ 	
                $[1.049,\ 1.053]$   &	3621$\ \pm\ $665&	1858$\ \pm\ $530&	76$\ \pm\ $14 \\ 	
                $[1.053,\ 1.057]$   &	3167$\ \pm\ $599&	1680$\ \pm\ $467&	76$\ \pm\ $14 \\ 	
                $[1.057,\ 1.061]$   &	3063$\ \pm\ $569&	1333$\ \pm\ $448&	70$\ \pm\ $15 \\ 	
                $[1.061,\ 1.065]$   &	3841$\ \pm\ $582&	685$\ \pm\ $461&	59$\ \pm\ $17 \\ 	
                $[1.065,\ 1.069]$   &	3343$\ \pm\ $439&	-45$\ \pm\ $324&	 - \\ 	
                $[1.069,\ 1.073]$   &	3377$\ \pm\ $525&	395$\ \pm\ $413&	59$\ \pm\ $21 \\ 	
                $[1.073,\ 1.077]$   &	2635$\ \pm\ $474&	684$\ \pm\ $368&	71$\ \pm\ $15 \\ 	
                $[1.077,\ 1.081]$   &	2632$\ \pm\ $426&	357$\ \pm\ $320&	64$\ \pm\ $18 \\ 	
                $[1.081,\ 1.085]$   &	2802$\ \pm\ $485&	647$\ \pm\ $377&	63$\ \pm\ $16 \\ 	
                $[1.085,\ 1.089]$   &	2121$\ \pm\ $421&	287$\ \pm\ $332&	74$\ \pm\ $18 \\ 	
                $[1.089,\ 1.093]$   &	2487$\ \pm\ $369&	-185$\ \pm\ $278&	 - \\ 	
                $[1.093,\ 1.097]$   &	2105$\ \pm\ $505&	1041$\ \pm\ $409&	68$\ \pm\ $15 \\ 	
                $[1.097,\ 1.101]$   &	2326$\ \pm\ $440&	100$\ \pm\ $355&	51$\ \pm\ $66 \\ 	
                $[1.101,\ 1.105]$   &	1962$\ \pm\ $369&	47$\ \pm\ $286&	44$\ \pm\ $137 \\ 	
                $[1.105,\ 1.109]$   &	1422$\ \pm\ $323&	216$\ \pm\ $246&	65$\ \pm\ $21 \\ 	
                $[1.109,\ 1.114]$   &	1420$\ \pm\ $453&	777$\ \pm\ $377&	63$\ \pm\ $17 \\ 	
                $[1.114,\ 1.118]$   &	697$\ \pm\ $377&	903$\ \pm\ $307&	73$\ \pm\ $17 \\ 	
                $[1.118,\ 1.122]$   &	1351$\ \pm\ $330&	234$\ \pm\ $257&	65$\ \pm\ $21 \\ 	
                $[1.122,\ 1.126]$   &	1373$\ \pm\ $297&	-60$\ \pm\ $229&	 - \\ 	
                $[1.126,\ 1.130]$   &	690$\ \pm\ $312&	340$\ \pm\ $255&	59$\ \pm\ $22 \\ 	
                $[1.130,\ 1.134]$   &	535$\ \pm\ $246&	130$\ \pm\ $197&	67$\ \pm\ $27 \\ 	
                $[1.134,\ 1.138]$   &	772$\ \pm\ $261&	205$\ \pm\ $199&	38$\ \pm\ $37 \\ 	
                $[1.138,\ 1.142]$   &	1246$\ \pm\ $266&	-71$\ \pm\ $200&	 - \\ 	
                $[1.142,\ 1.146]$   &	545$\ \pm\ $350&	456$\ \pm\ $298&	35$\ \pm\ $37 \\ 	
                $[1.146,\ 1.150]$   &	763$\ \pm\ $262&	206$\ \pm\ $205&	58$\ \pm\ $24 \\ 	
                \bottomrule\bottomrule
            \end{tabular}
        \end{center}
    \end{table}


    In addition, we also have tried to fit $S(980)$ with K matrix formalism (Appendix~\ref{app:a0_f0_K_matrix}) checked the significance of non-resonance $K^{+}K^{-}$ S wave in $S(980)$ (Appendix~\ref{app:a0_f0_nonRes}).


}

\subsection{Systematic Uncertainties}
\label{MIPWA-SYS}
\par{
    Systematic uncertainties taken in account:
    \begin{itemize}
        \item \uppercase\expandafter{\romannumeral1} Data-MC agreement for the BDTG output. 
            We apply the same BDTG as that in Sec.~\ref{MIPWA-BA} to a control sample, which passed the same event selection without the kinematic fit $\chi_{5c}^{2}$ cut criteria as that in Sec.~\ref{AASelection}, and compare the efficiency of data and MC.
            The efficiency of data (MC) is defined as $e_{data} = \frac{N_{d0}}{N_{d1}}$ ($e_{MC} = \frac{N_{M0}}{N_{M1}}$), where $N_{d0}$ ($N_{M0}$) and $N_{d1}$ ($N_{M1}$) are the number of events before and after applying the BDTG cut criteria.
            The comparison of efficiencies between data and MC is listed in Table~\ref{BDTG-SYS}.
            \begin{table}[htbp]
                \caption{The comparison of efficiencies between data and MC.}
                \label{BDTG-SYS}
                \begin{center}
                    \begin{tabular}{cccc}
                        \toprule\toprule
                        $m_{K^{+}K^{-}}$ (GeV/$c^{2}$) &  $e_{data} $ &  $e_{MC}$&  $e_{data} / e_{MC}$   \\
                        \hline
                        $[0.988, 1.014]$ &                0.3134\ $\pm$\ 0.0273 & 0.3069\ $\pm$\ 0.0062 & 1.0210\ $\pm$\ 0.0914 \\  
                        $[1.014, 1.018]$ &                0.3199\ $\pm$\ 0.0278 & 0.3641\ $\pm$\ 0.0068 & 0.8787\ $\pm$\ 0.0781 \\
                        $[1.018, 1.019]$ &                0.3968\ $\pm$\ 0.0357 & 0.4059\ $\pm$\ 0.0080 & 0.9775\ $\pm$\ 0.0900 \\
                        $[1.019, 1.021]$ &                0.3631\ $\pm$\ 0.0320 & 0.3970\ $\pm$\ 0.0083 & 0.9146\ $\pm$\ 0.0830 \\
                        $[1.021, 1.023]$ &                0.3704\ $\pm$\ 0.0330 & 0.3834\ $\pm$\ 0.0075 & 0.9662\ $\pm$\ 0.0881 \\
                        $[1.023, 1.029]$ &                0.3028\ $\pm$\ 0.0261 & 0.3190\ $\pm$\ 0.0064 & 0.9491\ $\pm$\ 0.0840 \\
                        $[1.029, 1.061]$ &                0.2932\ $\pm$\ 0.0254 & 0.2962\ $\pm$\ 0.0059 & 0.9900\ $\pm$\ 0.0878 \\
                        $[1.061, 1.150]$ &                0.2440\ $\pm$\ 0.0201 & 0.2502\ $\pm$\ 0.0049 & 0.9750\ $\pm$\ 0.0829 \\
                        \bottomrule\bottomrule
                    \end{tabular}
                \end{center}
            \end{table}
            We fit the shape of S(980) corrected with $\frac{e_{data}}{e_{MC}}$ and take the shift of $m_{0}$ and $\Gamma_{0}$ as the systematic uncertainty. The shift of $m_{0}$ and $\Gamma_{0}$ are 0.03 GeV/$c^{2}$ and 0.02 GeV, respectively.

        \item \uppercase\expandafter{\romannumeral2} Background subtraction. 
            We change the bin number and fit range and replace the background shape with a third-order Chebychev polynomial in the fit shown in Fig~\ref{MIPWA-ST} and take the shift as a resource of background ratio error.
            We vary the background ratio ( (3.8 $\pm$ 0.3)\% ) with one sigma and then take the largest shift of the fit of S(980) as the systematic uncertainty related to the background ratio. 
            %The shift of mass and width are 0.002 GeV/$c^{2}$ and 0.001 GeV/$c^{2}$, respectively.
            The background shape of generic MC is also replaced with that of ``Sideband" for data to perform a fit and the shift is taken as the systematic uncertainty related to the background shape.
            The shift of $m_{0}$ and $\Gamma_{0}$ are 0.002 GeV/$c^{2}$ and 0.001 GeV.

        \item \uppercase\expandafter{\romannumeral3} PID and tracking efficiency difference between data and MC. 
            To estimate the experimental effects related to the difference of PID and tracking efficiency between data and MC, based on the work~\cite{PID} and the work~\cite{Tracking}, we weight each event with the efficiency of data divided by that of MC and fit the shape of S(980).
            The shift of $m_{0}$ and $\Gamma_{0}$ are 0.001 GeV/$c^{2}$ and 0.013 GeV.

        \item \uppercase\expandafter{\romannumeral4} The $f_{0}(1370)$ remained in S(980). 
            We assume that the fit fraction (defined in Sec.~\ref{FF}) of $D_{s}^{+} \rightarrow f_{0}(1370)\pi^{+}$ is 5\% and then produced a MC sample with procedure $D_{s}^{+} \rightarrow f_{0}(1370)\pi^{+}$ to extract the shape of $f_{0}(1370)$ at the low end of $m_{K^{+}K^{-}}$ mass spectrum.
            We scale the number of $D_{s}^{+} \rightarrow f_{0}(1370)\pi^{+}$ according to the fit fraction of $D_{s}^{+} \rightarrow f_{0}(1370)\pi^{+}$ and the shape contributed from $f_{0}(1370)$ is shown in Fig.~\ref{f01370_SWave}.
            At last, the $f_{0}(1370)$ remained in S(980) is subtracted according to the scaled shape obtained above.
            The shift of $m_{0}$ and $\Gamma_{0}$ are 0.001 GeV/$c^{2}$ and 0.003 GeV.

            \begin{figure*}[htbp]
                \centering
                \mbox{
                    %\vskip -1.5cm
                    \begin{overpic}[width=0.48\textwidth]{plot/f01370_SWave.eps}
                    \end{overpic}
                }
                \caption{The shape of $f_{0}(1370)$ at the low end of $m_{K^{+}K^{-}}$ mass spectrum extracted from a MC sample with procedure $D_{s}^{+} \rightarrow f_{0}(1370)\pi^{+}$.}
                \label{f01370_SWave}
            \end{figure*}

        \item \uppercase\expandafter{\romannumeral5} Fit range. We vary the fit range from [0.988, 1.15] GeV/$c^{2}$ to [0.988, 1.145] GeV/$c^{2}$ and the shift of $m_{0}$ and $\Gamma_{0}$ are 0.002 GeV/$c^{2}$ and 0.003 GeV.  

    \end{itemize}

    %\uppercase\expandafter{\romannumeral1}
    %\uppercase\expandafter{\romannumeral2}
    %\uppercase\expandafter{\romannumeral3}
    %\uppercase\expandafter{\romannumeral4}
    %\uppercase\expandafter{\romannumeral5}

    All of the systematic uncertainties mentioned above are summarized in Table~\ref{MIPWA-Sys}.
    \begin{table}[htbp]
        \caption{Systematic uncertainties of partial wave analysis in the low $K^{+}K^{-}$ mass region.}
        \label{MIPWA-Sys}
        \begin{center}
            \begin{tabular}{cccc}
                \toprule\toprule
                Source   &                                                      $m_{0}$ ( GeV/$c^{2}$)  &$\Gamma_{0}$ ( GeV)\\
                \hline
                BDTG                     & 0.030                  &   0.020 \\
                Background subtraction   & 0.002                  &   0.001 \\
                PID and Tracking         & 0.001                  &   0.013 \\
                $f_{0}(1370)$            & 0.001                  &   0.003 \\
                Fit range                & 0.002                  &   0.003 \\

                \hline
                total                                   & 0.030                  &   0.024\\
                \bottomrule\bottomrule
            \end{tabular}
        \end{center}
    \end{table}
    So the result of $m_{0}$ and $\Gamma_{0}$ is:
    \begin{equation}
        \begin{array}{lr}
            m_{0} = (0.919 \pm 0.006_{stat} \pm 0.030_{sys}) \ {\rm GeV}/c^{2}, &\\
            \Gamma_{0} = (0.272 \pm 0.040_{stat} \pm 0.024_{sys})\ {\rm GeV}, &
        \end{array}\label{S-wave-sys} 
    \end{equation}

    %And we also perform a fit with two poles (Eq.~\ref{two-poles}) with two different widths.
    %\begin{equation}
    %    \begin{array}{lr}
    %        A_{S(980)} = \frac{K}{1 - iK}, &\\ 
    %        K = \frac{m_{0}\Gamma_{\alpha}}{ m_{\alpha}^{2} - m^{2}} + \frac{m_{0}\Gamma_{\beta}}{ m_{\beta}^{2} - m^{2}}, &
    %    \end{array}\label{two-poles} 
    %\end{equation}
    %where $\Gamma_{\alpha}$ and $\Gamma_{\beta}$ are the widths of the two poles, respectively.
    %And the result is:
    %\begin{equation}
    %    \begin{array}{lr}
    %        m_{0} = (0.922 \pm 0.006_{stat}) \ {\rm GeV}/c^{2}, &\\
    %        \Gamma_{\alpha} = (0.152 \pm 0.080_{stat}) \ {\rm GeV}/c^{2}, &\\
    %        \Gamma_{\beta} = (0.220 \pm 0.070_{stat}) \ {\rm GeV}/c^{2}. &
    %    \end{array}\label{S-wave2} 
    %\end{equation}
    %The result Eq.~\ref{S-wave2} will be taken in account in system uncertain of partial wave analysis related to model assumptions (Sec.~\ref{PWA-Sys}). 

}

